\section{Induction Coupler Behaviors}

For simplicity this discussion considers induction couplers comprised of magnets—whether permanent magnets or electromagnets—that can be characterized as single magnetic dipoles. Of particular interest here are induction couplers that include permanent magnets. Spinning permanent-magnet induction couplers consist of a mechanism with one or more permanent magnets spinning with variable speed about axes perpendicular to their dipole moments. The orthogonality of the dipole moment and the spin axis maximizes the change in magnetic field and therefore helps maximize the eddy-current effects. In its most elementary form, there is only a single dipole that spins in a plane. However, each spinning mechanism within the induction coupler can include any number of magnets. A motor spins this circumferential collection‎(or‎“array”)‎of‎magnets.

The more magnets in an array, the more uniform the magnetic becomes, smoothing the spatial variation around the exterior of the array. This smoothness decreases the change in the magnetic field as the magnets spin, which may be undesirable in an application that requires high eddy-current force. The number of magnets per array is one of the many considerations in the design of an induction coupler.

A straightforward example of an induction-coupler system is a single array spinning about an axis perpendicular to a flat, conductive surface. In this example, the coupler produces a force perpendicular to ω cross d where w is the angular velocity vector of the array relative to a plate-fixed reference frame, and d is the distance from the center of the dipole field to the surface.

Each array is associated with a single control degree of freedom. Its force and torque may project onto any of the six rigid-body degrees of freedom, depending geometries: the orientation of the coupler relative to the spacecraft’s‎center‎of‎mass,‎distance‎to‎the‎surface and the topography of the surface. . To first order, for a flat plate and a spin axis in a plane perpendicular to the surface normal, the effect is a force only, with no significant moment. The chaser spacecraft can use this force to torque the spacecraft if it is located so there is a moment arm between the spacecraft's center of mass and the array. Alternatively, two couplers can create a moment through a couple, which would be independent of‎ the‎ couplers’‎ position‎ relative‎ to‎ the‎ target’s‎ mass‎ center.

Experimental development of an induction coupler has shown that these forces vary with the magnitude and sign of ω and can produce milliNewton shear forces perpendicular to a surface for low power, using small COTS motors and permanent magnets. Specifically, two small motors driven by 12V at a 25% duty cycle while holding two neodymium permanent magnets (with an approximate dipole moment calculated to be around 8.5 *104 Am2) have together generated 5 mN, or 2.5 mN apiece. Spinning at 4200 RPM, each motor drew 0.25 amps of current, dissipating 0.75 Watts of electrical power. This power consumption corresponds to a power-specific force of 3.33 mN/Watt.

The angular velocity of the array determines the force because the induced voltage scales with ||v x B|| where v = ω x d is the relative velocity between the target conductor and the magnetic field. d is the vector position of the target relative to a point on the array’s‎ spin axis.

The direct scaling between force and ω makes two assumptions: that 1/ω is much larger than the characteristic time of the LR circuit that the conductor approximates and that the kinematics of the chaser and target are much slower than the period of rotation of the spinning magnetic fields that act on the target.
The force magnitude decreases approximately with 1/d4. Larger magnets increase the force (linearly with magnetic-moment magnitude). Greater thickness and conductivity of the target increase the induced current and thereby scale up the force.5‎ Spin speed tends to increase the force linearly by increasing the rate of change in the magnetic field, but increasing speed past a certain point shows diminishing returns because the so-called skin depth associated with induced current in the material drops with higher frequency. These trends provide the basis for designing an optimal induction coupler for a specific application.
