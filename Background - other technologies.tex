\documentclass{article}
\usepackage{amsfonts,amsmath,amssymb}
\usepackage{graphicx}
\usepackage{color}
\usepackage{url}



\begin{document}
\section{Background}
\subsection{Other Technologies}

There are other technologies that can produce
contactless forces between a spacecraft and a target.
Coulomb forces have been shown to produce useful
interactions between two charged spacecraft as long as
the distance between them is less than a Debye length. 
A number of different systems produce contactless
forces with magnetic interactions among controlled
dipoles on both the spacecraft and the target.All such approaches place requirements for specific hardware on both the chaser and the target (that is, the target must have launched with certain features already in place.) However, no spacecraft currently in orbit meet these criteria, to the
authors’ ‎knowledge.

Laser tweezers can produce contactless forces on an
uncooperative target.Error! Reference source not found.
However, the tweezers are best at manipulating micron- scale particles, a size restriction that no spacecraft beyond about TRL 1‎ can meet. Thruster plumes can also produce forces between a spacecraft and a target. However, typically the combustion products from thrusters carry significant risk of contaminating optical instruments and solar panels, among other disadvantages. We conclude that current technology is limited to direct mechanical contact as the only other option to create forces on a target that has not been designed for an interactive mission.
\bibliographystyle{plain}
\bibliography{/bibliography/2d_inspection_bib.bib}
\end{document}