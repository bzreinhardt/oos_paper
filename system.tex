\section{Induction Coupler System}

The rotational speed of each array controls the force between that array and the surface of the target. The sum of the forces from all the arrays and their resultant torques can be mapped through the linearized dynamics of the target and chaser to plan control inputs based on a desired maneuver. Nonlinearities are likely best accommodated through gain scheduling, a topic that the authors intend to take up as future work.
The present analysis assumes that the inspection spacecraft maintains a small constant distance from the surface. With an operating range of a few centimeters, the surface for most of the ISS looks like an infinite plane to the inspection spacecraft. The analysis assumes that the proposed inspection vehicle conforms to the cubesat standard so the mass of the entire system is not above m = 4 kg.
An induction coupler with only one or two spinning arrays can achieve planar motion. The motion is nonholonomic; i.e. the chaser’s ability to move in a given direction depends on its orientation.
Three spinning arrays can drive three independent planar degrees of freedom. In practice, more should be used for redundancy and greater control authority. As long as the arrays have sufficient spatial separation, their forces simply superimpose. For a particular application, these principles inform tradeoffs among force, power, mass, and the reliability of moving parts introduced by the additional arrays.
In the induction coupler, each array is the located at some radius, r, from the spacecraft's center of mass. Each array has a spin axis  oriented parallel to the target surface at an angle φ‎ from the x-axis in spacecraft coordinates. See Figure 5: A single-magnet induction coupler.