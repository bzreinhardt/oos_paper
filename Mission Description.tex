\section{Mission Description}

A small inspection spacecraft can use induction couplers to crawl along the surface of a larger target spacecraft without physically grappling the target. The inspector can fulfill a number of functions including investigating problematic areas, scanning for damage, performing small tasks, or providing support for astronauts on space walks. This paper focuses on the International Space Station (ISS), but a similar inspection spacecraft could enable unique OOS missions to inspect and repair other large satellites.

The inspection spacecraft would use induction couplers to pull itself along the aluminum surface of the ISS, maintaining a separation distance of a few centimeters. From this vantage point, it could act like a damage- inspection Roomba, canvassing the surface and automatically detecting damage from micrometeorite strikes. It could also be controlled directly by an astronaut in the space station to look at a problem area and possibly use an attached robotic arm to perform small actions. During spacewalks, the spacecraft could act like an extra pair of hands, remaining near an astronaut and holding tools. Robotnaut and the Canada Arm have demonstrated the value of this role, but are limited by rails to specific locations.
The ability to traverse the exterior of the ISS without being constrained to travel on rails, attach to specific hard points, or manage a finite propellant supply can
Reinhardt
free up one of the most valuable resources on the ISS - astronaut time.

