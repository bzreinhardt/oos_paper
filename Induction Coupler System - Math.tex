\documentclass{article}
\usepackage{amsfonts,amsmath,amssymb}
\usepackage{graphicx}
\usepackage{color}
\usepackage{url}

\bibliographystyle{plain}

\begin{document}


For N arrays in this special orientation, the transformation between the angular speeds of the arrays and the net force and torque on the spacecraft is:
\begin{equation}\label{eq:forces}
 \begin{bmatrix} F_x\\F_y\\\tau \end{bmatrix} = C\begin{bmatrix} \cos{(\phi_1 -\pi/2)} & \cos{(\phi_2 -\pi/2)} & ... & \cos{(\phi_N -\pi/2)} \\
\sin{(\phi_1 -\pi/2)} & \sin{(\phi_2 -\pi/2)} & ... & \sin{(\phi_N -\pi/2)} 
\\\textbf{r}_1\cdot \hat{\textbf{a}}_1 & \textbf{r}_2 \cdot \hat{\textbf{a}}_2 & ... & \textbf{r}_N \cdot \hat{\textbf{a}}_N \end{bmatrix}
\begin{bmatrix} \omega_1\\\omega_2\\...\\\omega_N \end{bmatrix}
\end{equation}

In \ref{eq:forces}, C is a matrix of constants that take the angular velocities to forces and torques. These constants will vary with ‎the‎ distance ‎to‎ the ‎target ‎and ‎the ‎target’s‎ properties.
The Jacobian J is the geometry-dependent matrix from \ref{eq:forces}:
\begin{equation}\label{eq:sfJacobian}
J = \begin{bmatrix} \cos{(\phi_1 -\pi/2)} & \cos{(\phi_2 -\pi/2)} & ... & \cos{(\phi_N -\pi/2)} \\
\sin{(\phi_1 -\pi/2)} & \sin{(\phi_2 -\pi/2)} & ... & \sin{(\phi_N -\pi/2)} 
\\\textbf{r}_1\cdot \hat{\textbf{a}}_1 & \textbf{r}_2 \cdot \hat{\textbf{a}}_2 & ... & \textbf{r}_N \cdot \hat{\textbf{a}}_N \end{bmatrix} 
\end{equation}
The speed-force Jacobian \ref{eq:sfJacobian}  imposes constraints on the design of the induction coupler.

\begin{itemize}
\item  $ \textbf{r}_i \cdot \hat{\textbf{a}}_i $ must be nonzero for at least one array, corresponding to the requirement that all of the spin axes cannot be perpendicular to their moment arms, nor can all $r_i$ be zero (i.e. some of the arrays cannot be at the unlikely location of the target spacecraft's center of mass).
\item  Not all of the spin axes ai can be parallel. If they are, all $\phi_i$ to be integer multiples of $\pi$ radians ‎apart, and the Jacobian would not be full rank.
\end{itemize}

The specific case of an inspection vehicle equipped with an induction coupler consisting of three magnet arrays with $\Vert r \Vert = 10 \text{cm}$ helps fix ideas.
\begin{equation}\label{eq:specificPhi}
\mathbf{\phi} = \begin{bmatrix} \frac{3\pi}{4}&\frac{pi}{2}&\frac{pi}{4}\end{bmatrix}
\end{equation}
and
\begin{equation}\label{eq:specificR}
\hat{\textbf{r}} = \begin{bmatrix} \frac{\sqrt{2}}{2} & 0 & -\frac{\sqrt{2}}{2} \\
\frac{\sqrt{2}}{2} & 1 & \frac{\sqrt{2}}{2}\end{bmatrix}
\end{equation}

The jacobian is: 

\begin{equation}\label{eq:specificJac}
\begin{bmatrix} \frac{\sqrt{2}}{2} & 1 & \frac{\sqrt{2}}{2} \\
\frac{\sqrt{2}}{2} & 0  & -\frac{\sqrt{2}}{2}
\\0  &  1 & 0 \end{bmatrix}
\end{equation}

This Jacobian ignores the constants scaling $ F_x $, $ F_y $, and‎ $\tau$‎ based on the system state and physical constants of the target.
An induction coupler of this type on a 4 kg spherical spacecraft of radius 0.1m, can generate a linear acceleration of about 
$2*10^-3 \frac{m}{s^2}$ 
and an angular acceleration of 
$9*10^-3 \frac{rad}{s^2}$. 
These values, while modest, allow an inspection vehicle to overcome perturbation forces and move slowly over the surface of the ISS to inspect it. These values are based on a small prototype comprised of COTS components and an induction coupler with only three magnet arrays. Using additional arrays and optimizing the hardware are two straightforward ways to increase the capabilities in this example.

\end{document}